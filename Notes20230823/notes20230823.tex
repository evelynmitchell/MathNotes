\documentclass[]{scrartcl}

%opening
\title{Notes 2023-08-23}
\author{Evelyn Mitchell}

\usepackage{amsmath,amssymb}

\DeclareRobustCommand{\bbone}{\text{\usefont{U}{bbold}{m}{n}1}}

\DeclareMathOperator{\EX}{\mathbb{E}}% expected value

\usepackage{glossaries} \makeglossaries

\begin{document}

\maketitle

\begin{abstract}

\end{abstract}

\section{Mathematical Truth}

https://plato.stanford.edu/entries/intuitionistic-logic-development/\#ProoInte

Truth in intuitionistic logic is \textbf{by construction}. That is, you must be able to construct an instance of the asserted statement in order to assume it.

\begin{quote}
BHK-Interpretation (for "Brouwer, Heyting, Kolmogorov") or Proof Interpretation as given by Troelstra and van Dalen in Constructivism in Mathematics (Troelstra and van Dalen 1988: 9):

(H1)A proof of $A \land B$ is given by presenting a proof of $A$ and a proof of $B$. 

(H2)A proof of $A \lor B$ is given by presenting either a proof of $A$ or a proof of $B$ (plus the stipulation that we want to regard the proof presented as evidence for $A \lor B$). 

(H3)A proof of $A \implies B$ is a construction which permits us to transform any proof of $A$ into a proof of $B$. 

(H4)Absurdity $\bot$ (contradiction) has no proof; a proof of $\not A$ is a construction which transforms any hypothetical proof of A into a proof of a contradiction. 

(H5)A proof of $\forall xA(x)$ is a construction which transforms a proof of $d \in D$ ($D$ the intended range of the variable $x$) into a proof of $A(d)$. 

(H6)A proof of $\exists xA(x)$ is given by providing $d \in D$, and a proof of $A(d)$.
\end{quote}

\section{typesetting}

For proofs: $https://www.overleaf.com/learn/latex/Theorems_and_proofs$

$\land$ is logical and.

$\lor$ is logical or.

$\bot$ is contradiction.

$\implies$ is logical implication. If the left side is true, the right side is true.

$\in$ is an element of

$\forall$ for all

$\exists$ exists


\section{Glossary}

\newglossaryentry{condition}{%
name={condition},
description={%
	To \textbf{condition} a formula is to give a proposition or propositions which restrict the formula to certain values.
},
} 

%\gls{label}. 
\gls{condition} Example: 

$q(s,a) = \EX[G_t |S_t=s, A_t=a]$

The Expectation is conditioned on $S_t=s$ and $A_t=a$ .

(The example is from Reinforcement Learning: https://youtu.be/TCCjZe0y4Qc ; Sutton and Barto 2018)
 
\newglossaryentry{contradiction}{%
name={contradiction},
description={\textbf {contradiction}, symbol $\bot$ . Also known as Absurdity.
}
}

\printglossaries

\section{Bourbaki Theory of Sets}

I'm taking a suggestion to follow the formal path into proof writing. Thus, Nicolas Boutbaki, Elements of Mathematics, Volume 1: Theory of Sets

Each chapter proceeds in a logical order, with later parts only requiring knowledge of what has come before, so no backtracking, or assuming knowledge of a later topic.
The chapters start with \textbf{definitions}, the \textbf{axioms} and the \textbf{theorems} of the chapter.

Additional results are variously named as "propositions", "lemmas", "corollaries",
 "remarks" and so on. Commentaries on especially important theorems are called "scholium".

Dangerous bend passages are marked with a large Z in the margin. (So that is whewrwe Knuth got it!). These mark warnings to the reader of serious errors.

Exercises are designed to cement learning, and to prompt understanding in the reader of results which are interesting, but didn't fit in the text.

The authors are creating a self-contained text, as an explicit goal. There are Historical Notes with pointers to the rest of the literature, including unsolved problems.

This volume includes:
1. Description of formal mathematics.

2. Theory of sets.

3. Ordered sets; cardinals; natural numbers.

4. Structures.

\subsection{Introduction}

\end{document}
