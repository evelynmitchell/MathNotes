\documentclass[]{scrartcl}

%opening
\title{Notes 2023-08-25}
\author{Evelyn Mitchell}

\usepackage{amsmath,amssymb}
\DeclareRobustCommand{\bbone}{\text{\usefont{U}{bbold}{m}{n}1}}

\DeclareMathOperator{\EX}{\mathbb{E}}% expected value

%\usepackage{glossaries} \makeglossaries

\makeindex 

%\showindex

\begin{document}

\maketitle

\begin{abstract}

\end{abstract}

\section{Bourbaki Theory of Sets}

\subsection{Introduction}

Continuing from the section break on page 12.

Consistency or non-contradiction as a major question for logicians. Mathematical languages should be non-contradictory, that is, they should be consistent. 

To be non-contradictory either a statement can be true, or it's negation, but not both at the same time. (https://en.wikipedia.org/wiki/Law\_of\_noncontradiction) \index{noncontradiction}

A related concept is The Law of the Excluded Middle

 \(https://en.wikipedia.org/wiki/Law\_of\_excluded\_middle\).
  \index{excluded middle}

The difference between the two ideas is that non-contradiction slices possible states of the the world into two, one where $p$ is true and one where $\neg p$ is. The Law of Excluded Middle says at least one of "$p$ is true" or "$\neg p$ is true". 

Non-contradiction gives us mutual exclusion. Excluded Middle gives us exhaustive covering.


\begin{quote}
	The law of non-contradiction and the law of excluded middle create a dichotomy in "logical space", wherein the two parts are "mutually exclusive" and "jointly exhaustive". The law of non-contradiction is merely an expression of the mutually exclusive aspect of that dichotomy, and the law of excluded middle is an expression of its jointly exhaustive aspect. - Wikipedia Law of Noncontradiction, retrieved 2023-08-25
	\end{quote}

Bourbaki uses metamathematics to examine consistency with metamathematical methods. 

\begin{quote}
	To say that a theory is contradictory means in effect that it contains a correct formalized proof which leads to the conclusion $0 \ne 0$ . Now, metamathematics can attempt, using methods of reasoning borrowed from mathematics, to investigate the structure of such a formalized text, in the hope of "proving" that such a text cannot exist. p 12
\end{quote}


\end{document}
